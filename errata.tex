%!TEX program = xelatex
\documentclass[lang=zh,11pt,numbers]{errata}
\usepackage{amsmath}
\usepackage{amssymb}
\usepackage{bm}

\title{視覺SLAM十四講:從理論到實踐(第二版) \\ 勘誤表}
\author{高翔}
\date{\today}

\begin{document}
\maketitle
\section{勘誤表說明}
由於能力所限,書籍內容難免有一些錯誤,我們對此表示抱歉。在這個文檔裏,我們列出自第一次印刷(2019年8月)後書中的錯誤。大多數錯誤應該會在下一次印刷中修復。所以,請讀者留意扉頁中的印刷次數,並對照本文檔進行斟別。

對於代碼的改動,請參照當前的github頁面,勘誤表僅針對書籍內容。本勘誤表僅對第二版書籍有效,第一版書的勘誤請參考第一版書對應的github頁面。

由於排版可能在不同印刷次數中存在微小改動,勘誤表中的頁面和段落等標記,僅針對當前次印刷有效。

如果您認爲書中內容存在錯誤,請給我發郵件,或者通過出版社提供的反饋通道發送信息。直接給我發郵件會比較方便。
\section{第一次印刷(2019年8月)}
\begin{table}[!htp]
	\centering
	\caption{第一次印刷勘誤}
	\begin{tabular}{c|cccc}
		\hline\hline
		序號 & 位置 & 改動前 & 改動後 & 說明 \\\hline
		1 & 彩頁1右上圖標題 & 拓撲地 & 拓撲地圖 & 美編加工時漏字 \\
		2 & 式3.41 & $
		\begin{aligned}
		\theta &= \arccos(\frac{\mathrm{tr}(\bm{R}-1)}{2}) \\
		&=\arccos(2s^2-1).
		\end{aligned}$ & $
		\begin{aligned}
		\theta &= \arccos(\frac{\mathrm{tr}(\bm{R})-1}{2}) \\
		&=\arccos(2s^2-1).
		\end{aligned}$ & $\mathrm{tr}$括號位置有誤 \\
		3 & P188 終端輸入 & d1.png d2.png & 1\_depth.png 2\_depth.png & 深度圖文件名 \\
		4 & 參考文獻61 & P3p(blog) $\ldots$ & 刪除 & 網站已過期,網址不可訪問 \\
		5 & P326 式(12.14) & $d_C {\bm{P}_C} = \ldots .$ & 刪除最右側$\bm{t}_{\mathrm{RW}}$前的$\bm{K}$ & 多一個$\bm{K}$ \\
		6 & P156 & 金字塔是計算圖視覺中 & 計算機視覺中 & 錯別字 \\
		\hline\hline
	\end{tabular}
\end{table}

\section{第四次印刷(2019年10月)}
\begin{table}[!htp]
	\centering
	\caption{第四次印刷勘誤}
	\begin{tabular}{c|cccc}
		\hline\hline
		序號 & 位置 & 改動前 & 改動後 & 說明 \\\hline
		7 & P265 第10講主要目標 & 第3條 & 刪去 & 正文內未介紹IMU(計劃但沒有實裝) \\
		\hline \hline
	\end{tabular}
\end{table}

\section{一些不在勘誤表內的改動}
除上述改動之外,還有一些不在書籍文本內的改動,需要向讀者說明。大部分這裏的勘誤來自於github issues.

\begin{enumerate}
	\item 第8講的直接法實現,需要使用OpenCV 4支持的cv::parallel\_for\_函數。如果讀者使用較舊的版本,需要對代碼做一些改動。具體的改動方法請參照對應版本的OpenCV文檔,或者參考~\url{https://github.com/gaoxiang12/slambook2/issues/32}。
	\item 第4講的examples/trajectorError例子中,構造SE3的部分四元數應該使用$w,x,y,z$順序。
\end{enumerate}

\end{document}
